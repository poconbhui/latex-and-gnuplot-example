\documentclass{article}

\usepackage{graphicx}
\usepackage{verbatim}
\usepackage{parskip}

\begin{document}


\section{Using The Plot}

The gnuplot generated .tex file is included in a caption after this paragraph
using the code
%
\begin{verbatim}
\begin{figure}
    \input{plot/plot}
    \caption{Our lovely gnuplot plot!}
\end{figure}
\end{verbatim}
%
Note the absence of ``src/'' in the filename specification.
The ``src'' directory has been included on the command line.


The plot.tex file then has the line
%
\begin{verbatim}
\includegraphics{src/plot/plot}
\end{verbatim}
%
since the output file was specified with a relative name from
the root directory.
Since pdflatex is then also called from the root directory, it can find
files in the ``src/plot'' path.
The file ``src/plot/plot'' could refer to either the .pdf or .eps file.
Since pdflatex can't handle .eps files, it loads the .pdf file.


% Include the damn plot already!
\begin{figure}
    \input{plot/plot}
    \caption{Our lovely gnuplot plot!}
\end{figure}


\end{document}
